\documentclass[a4paper,]{article}
\usepackage[left=2.5cm,top=2.5cm,right=2.5cm,bottom=3cm]{geometry} 
\usepackage[utf8]{inputenc}
\usepackage{graphicx} % Required for inserting images
\usepackage[table,xcdraw]{xcolor}
\usepackage{lipsum} % for dummy text
\usepackage{mathdots} % para el comando \iddots
\usepackage{mathrsfs} % para formato de letra
\usepackage{marvosym}
\usepackage{amsmath}
\usepackage{amsfonts}
\usepackage{amssymb}
\usepackage{listings}
\usepackage{caption}
\usepackage{float}
\usepackage{markdown}


\title{\textbf{HW4: Introduction to Financial Engineering}}
\author{Miguel Angel Aguilo Gonzalez, 1699413 \\ Judit de Paz Ramírez, 1570590 \\ Laia Mòdol Rodríguez, 1565282 \\ Elena Rubio Zabala, 1699049 \\ Guillem Tutusaus Alcaraz, 1533701 } 
\date{Noviembre 2023}

\begin{document}

\maketitle
\newpage

\section*{Ejercicio 1}
Vamos a programar un algoritmo de árbol binomial para calcular los precios de las opciones. \\

Como todos sabemos, las variables necesarias para construir un árbol son el precio de las acciones $S$, el movimiento ascendente permitido $u$ y el movimiento descendente permitido $v$, dónde $0<v<1<u$. También necesitamos la variable $N$ que es el número de pasos de tiempo que se utilizarán.\\

Para crear una función que nos devuelva los posibles precios de las acciones para intervalos de tiempo sucesivos $t\in(0,N)$, cuando suministramos el precio inicial de estas, y el número de intervalos de tiempo que queremos que se calculen, empezamos creando una matriz. Ésta tiene dimensiones $(N+1)\times(N+1)$, dónde las filas representan la profundidad del árbol y las columnas representan los nodos (escenarios para el precio de las acciones). Inicializamos la matriz con ceros, que es donde se encuentra el precio inicial de las acciones $S$. \\

Utilizamos dos bucles anidados (\texttt{for}) para recorrer las filas y las columnas de la matriz, llenándola con los precios de las acciones. En cada posición $M[i, j]$, se calcula el precio de la acción utilizando la fórmula $$S \cdot u^{(j - 1)} \cdot v^{(i - j)}$$ y se almacena en esa posición. \\

La función que obtenemos es
\vspace{1cm}
% código función en Markdown

Finalmente, podemos observar que para cualquier precio de acción en el momento $t-1$, supongamos $M[t,j]$, nuestra matriz aplica una caída de precio $v$ en la siguiente fila manteniendo la columna, que corresponde a la posición $M[t + 1,j]$, mientras que los aumentos se colocan en la posición $M[t + 1,j + 1]$. Por lo tanto, hemos creado una matriz que se caracteriza por ser una triangular superior la cual tendrá $t+1$ elementos distintos de 0 por fila. \\

Calculemos ahora la probabilidad neutral al riesgo de un momento ascendente $q$. Para ello necesitamos los valores $u$, $v$ y los tipos de interés $r$ con el paso de tiempo $dt$ (recordemos que la probabilidad de un movimiento a la baja será $(1-q)$). Recordemos también que mediante un argumento sin arbitraje tenemos
$$S(1+rdt) = quS+(1-q)vS$$

Asumiendo $S\neq 0$ podemos aislar $q$ de la ecuación que relaciona los precios futuros y la probabilidad neutral al riesgo en un modelo binomial. Tenemos
$$S(1+rdt) = quS+(1-q)vS \Longleftrightarrow 1+rdt-v=q(u-v)\Longleftrightarrow q=\frac{1+rdt-v}{u-v}$$

Así pues, podemos definir la función mediante el siguiente código: 
\vspace{1cm}
%código 

A continuación queremos escribir una función que calcule el valor de una opción de forma recursiva. El primer paso a dar es crear un árbol vacío que rellenaremos con el valor de la opción de forma recursiva. Por lo tanto, creamos una matriz de dimensión $(N+1)\times (N+1)$ dónde almacenaremos los valores de las opciones en cada nodo del árbol, así pues, debe tener las mismas dimensiones que la matriz triangular $M$ creada anteriormente.\\

Luego llenaremos los últimos nodos del árbol con la función de pago (la función debe aceptar una llamada (\textit{CALL}) o una opción de venta (\textit{PUT})), así tendremos 
$$P[N+1,j]=\mbox{\texttt{payoff}}(K, M[N+1,j])$$
donde \texttt{payoff}$(K,S)$ es la opción con pago de \textit{strike} $K$, cuando el precio final de las acciones es $S$. Definimos pues las funciones:
%código call and put
\vspace{1cm}

Ahora usamos un bucle para completar los nodos un paso hacia atrás usando la fórmula
$$V(t)=\frac{V(t+dt)^{+}q+V(t+dt)^{-}(1-q)}{(1+rdt)}$$
dónde $V(t+dt)^{+}$ significa el valor de la opción en $t+dt$ en el nodo ascendente, y de manera similar para $V(t+dt)^{-}$.
%código 
\vspace{1cm}

El precio de la opción que buscamos es $V(0)$ y, por lo tanto, será la posición $P[1,1]$ de la matriz que imprime nuestra función creada.\\

En resumen, hemos creado un modelo binomial en una matriz llamada opción binomial de valor. Hemos observado que el elemento $[1,1]$ de la matriz corresponde al primer componente de $V(t)$ y, por lo tanto, al precio final de una opción. De esta manera, para obtener este valor podemos crear la siguiente función:
%código


\section*{Ejercicio 2}
Usaremos las funciones que hemos definido anteriormente para calcular el precio de una opción \textit{CALL} con \textit{strike} ATM (\textit{At The Money}, es decir, el precio de negociación subyacente actual), con una subyacente que cotiza a USD 100 y vence en un año, por lo tanto, tenemos valores de $S=K=100$ y $N=12$. \\

Nos fijaremos un paso de tiempo mensual ($dt=1/12$) y una tasa de interés anual del 10\%, así $r=0,1$. Finalmente, sabemos que la acción puede subir o bajar un 1\% cada mes, dónde tenemos $u=1,01$ y $v=0,99$. \\

Ejecutando el siguiente comando
%código
\vspace{1cm}

obtenemos un resultado de 9,478797, que es el precio de la opción \textit{CALL} que queríamos obtener.

\end{document}
